\newpage
\section{Installation}
The installation will be carried out on Linux system.
\subsection{Requirements}
\begin{itemize}
    \item CALM source archive -- \url{https://github.com/wozniczu/CALM.git}
    \item Cmake -- \url{https://cmake.org}
    \item C++ compiler (for example gcc)
    \item CERN Root Framework -- \url{https://root.cern/install/}
\end{itemize}
Cmake, compiler and Root have to be installed on the machine and CALM source archive extracted to the installation directory.
\subsection{CALM archive}
The archive contains:
\begin{itemize}
    \item Source code of CALM -- "include" and "src" directories
    \item Particle database that are used by CALM in its runtime -- "Shared/particles.data"
    \item Configuration files -- event.ini, config.ini
    \item Calibration files -- "distributions" directory
    \item Script that helps to launch CALM -- "run\_CALM.sh"
\end{itemize}
Details and utility of configuration files and script will be described later inside this manual.
\subsection{Compilation}
Type inside installation directory:\\\tab mkdir build \\\tab cd build \\\tab cmake .. \\\tab make
\subsection{How to launch CALM}
To launch CALM with default configuration just run "calm" file inside build directory.\\
Alternatively, run "run\_CALM.sh" to generate multiple events simultaneously (runs the CALM program on multiple cores).